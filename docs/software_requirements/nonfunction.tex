\section{Other Nonfunctional Requirements}
% Nonfunctional requirements are shall-statements about how the software
% performs or is written.  It is not a statement about what the software does.
% For example, the function requirements for a Fibonacci function is this: the
% function shall return the nth Fibonacci number when provided n as input.  A
% nonfunctional requirement would be these: the function shall take time O(n)
% and all lines of the software shall be reachable by some test-case.
%

\subsection{Performance Requirements}
% Your project will have performance requirements.  How long is the user willing
% to wait for the different features to execute?   For example, the software
% shall authenticate authorized users withing 250ms.   

The software shall import data within 250ms.
The software shall create reports and graphs within 250ms.
The software shall generate the graphical user interface within 300ms.

\subsection{Safety Requirements}
% It is unlikely that your project will have any safety requirements.  This
% section would be used for software that controls a physical device that could
% potentially cause harm.  

The software shall not corrupt the data that is provided as input to it.

\subsection{Security Requirements}
% Do not overlook this section.  Consider the three primary topics of computer
% security: Confidentiality, Integrity, and Availability.   Your project will
% have security requirements. 
Given the projected nature of our product being a local executable, we will not interface with any outside sources except for the files being input to the program to be read. This will protect the confidentiality of the data. Our program will not alter the files being read, so as to protect the integrity of the data, and since we are creating a local executable, the program and its data will be readily available for authorized users.

\subsection{Software Quality Attributes}
% Quality requirements can be the most difficult to write.  The desired quality
% must be written in a meaningful and testable way.  For example, All functions
% within the software shall have a cyclomatic number less than 10.  

All functions within the software shall have a cyclomatic number less than 6.