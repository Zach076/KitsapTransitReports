\section{Introduction}
% This section is used for multiple kinds of documents.  It describes the
% purpose of this document in the context of the rest of the product's
% documentation library.  

\subsection{Purpose}
% What is the purpose of this document as opposed to the vision and scope,
% software design document (492), testing plan (493), et al.

This document is intended to serve as a reference for all of the requirements of our software. It will describe the input requirements and format in order to run the software. It will also describe what the software is capable of doing in terms of features and include a glossary of terms that pertain to this software.

\subsection{Document Conventions}
% Describe the conventions used in this document.  For example, throughout this
% document, the use of the plural shall imply the singular unless otherwise
% stated.  Now you can avoid parenthetical plurals like student(s).   

Throughout this document, the reference to data provided as input shall imply data in the file format .xlxs or of .csv format.

\subsection{Intended Audience and Reading Suggestions}
% Formally state the intended audience for this document.  For the SRS, the
% audience is usually developers, quality control, and documentation.  You are
% free to describe whatever readers you feel are appropriate, but you should not
% describe the reader in terms of the class.  That is, do not refer to a teacher
% or other students.  

The intended audience of this document would be any developer or user of the software, who wants to gain a better understanding of what is needed in order to use the software. It is also intended for anyone who wants to gain a better understanding of the purpose and capabilities of the software.

\subsection{Project Scope}
% Put this developed project in the context of the overall product.  This is a
% brief summary of the vision and scope.  Just enough for the reader to
% understand the context for the requirements in later sections.

The software that we will create, will read in excel files of type .xlsx or of CVS format. Then it will parse the data and generate a graphical user interface that will allow the user to interact with the collected data in a manner that allows for easy accessibility of all of the data.

%\subsection{References}
% List other documents that you refer to in the rest of this document.  Include
% unreferenced, but important documents for the project.