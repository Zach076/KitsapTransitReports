\section{Features}
% This section contains a list of requirement statements.  This should be of the
% form "the system shall..."  and worded in a way that satisfaction of the
% requirement can be verified.  For example, the software shall authenticate
% users with at least two identification factors.  

The software shall produce a graphical user interface that allows the user to interact with the inputed data.

% [For Each Feature]
\subsection{Name of feature}

Graphical User Interface

\subsubsection{Description}
% Describe the feature and how it fits into the overall product.  

The graphical user interface is what the user will see after they input their data. It will be in addtion to our main feature of creating reports of the inputed data.

\subsubsection{Priority}
% Describe the relative importance of this feature. 

This feature is important to have since it is what the customer requested, but is not the most important feature fo the software becasue there are other ways to display the collected data.

\subsubsection{Stimulus and Response}
% What event will trigger the feature and how should the system respond.  This
% is probably an excerpt of a use case. 

The user will provide their data as input to our software. Then our software will create reports with that data. After that the graphical user interface will be created, so that the user can access and interact with the data. 

\subsubsection{Functional Requirements}
% Formally state the functional requirement.
% The low-level format command shall require authorization by two
% system-administrators before beginning the low-level format operation.

The software shall require data to be provided as input in order for the graphical user interface to be generated.
