\section{Overall Description}
% Describe the product's context in the larger business or industry setting.  Do
% not include specific features.  Give the reader an understand of how those
% features fit into the larger setting. 
%

\subsection{Product Perspective}
% How does the product fit in the business larger processes.  How is the user
% intended to fit the software into their business activities?   Consider
% including a figure that illustrates these relationships. 

\subsection{Product Features}
% This is a list of high-level description of the functional behavior of the
% product.  This should give the reader a better understanding of how the formal
% requirements fit together. 

\subsection{User Classes and Characteristics}
% Describe the different rolls or classes of users.  For each user class,
% describe the user class's principal characteristics.  For example, unix
% systems have at least two classes of users: system administrators and
% operators.  
There is only one user class for our product whose only purpose is to load data files into the program and use the output of reports. 

\subsection{Operating Environment}
% What is the expected environment?  For example, the product could be a desktop
% application with users who work in a formal office environment.  Contrast this
% with a mobile application for mountain biking that keeps track of GPS locations.
Our product is a simple desktop application which we expect to be run on windows, though we plan to code in Java which could be run on any major operating system. 

\subsection{Design and Implementation Constraints}
% List an constraints that are part of the project.  For example, health
% services applications must implement HIPAA regulations. 
From our current understanding, our only implementation constraint is that we must be able to read and parse data from .xls, .xlsx, and .csv files. 

\subsection{Assumptions and Dependencies}
% List assumptions and dependencies that are not formal constraints.  Items in
% this list will, if changed, will cause a change in the formal requirements in
% the next section.
Our team expects to be using .xls, .xlsx, and .csv files as input for our program, we expect a Java based program to be acceptable to our client, and we expect that all requirements for the product are outlined in the 'Product Features' section.

