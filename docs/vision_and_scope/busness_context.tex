\documentclass{article}
\usepackage[utf8]{inputenc}

\begin{document}

\section{Business Context}

\subsection{Stakeholder Profiles}
% This section describes who will benefit from the product both directly and
% indirectly.  Identify stake holders by their roll and describe their needs and
% interactions for that role.
The stakeholders in the scope of this project include the employees of Kitsap Transit
and those who use Kitsap Transit services. Some employees of Kitsap Transit will be
working directly with our product in an effort to analyze bus route data and potentially
modify bus routes or the amenities of certain bus stops. By this, those who use Kitsap
Transit services may see changes due to the data we present in the product.Employees will
need a clear and easy to use product to show data in a meaningful way.

\subsection{Project Priorities}
% Prioritize the major features identified above based on the relative
% return-on-investment.  
The features which we need to prioritize the most is how to import the data given. We would
like to make that as simple a process we can without detracting time from other important
features, such as displaying the data in a meaningful way. Deciding how to display the data
will probably be the most important design decision in this project so we plan to get that
done as soon as possible. We want to make sure that we can design something Kitsap Transit
is happy with. Once we do that, all we need to do is build it. We will build back-end first
and then the front-end.

\subsection{Operating Environment}
% Describe how and where the product will be used.  
The product will be used by the employees of Kitsap Transit to assess the passenger data.
We expect the product to be used on Windows PC's and we plan to build the product on Java
which supports other major operating systems as well.


\end{document}
